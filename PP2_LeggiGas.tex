%==================================================
%      PREAMBOLO e DICHIARAZIONI INIZIALI
%==================================================
\documentclass[10pt,oneside,a4paper]{article}

\usepackage[latin1]{inputenc} 
\usepackage[italian]{babel}
\usepackage{siunitx} %Inserisce automaticamente i dati con le unit�  di misura correttamente formattate del SI (utilizzo: \SI{0.82}{m^2}, in generale \SI{misura con il punto decimale}{unit�  di misura})
\sisetup{output-decimal-marker = {.}, separate-uncertainty = true, input-uncertainty-signs = \pm, detect-weight=true, detect-family=true} %per usare SI con il punto decimale
\usepackage{listings} %Per citare codice informatico formattandolo correttamente
\usepackage{amsmath}
\usepackage{graphicx}
\usepackage{geometry}
\usepackage{epigraph}
\usepackage{booktabs}	%tabelle migliorate
\usepackage{tablefootnote}	%note a pi� di pagina in tabella
\usepackage{threeparttable} %tabella con note a pi� di tabella
\usepackage{caption}	%descrizione per figure
\captionsetup{tableposition=top,figureposition=bottom,font=small} %setup descrizione
\usepackage{float}
\usepackage{esvect} %vettori
\usepackage{longtable} %tabelle lunghe
\usepackage[dvipsnames]{xcolor}
\definecolor{sepia}{HTML}{80002A}
\usepackage[colorlinks=true, citecolor=black, linkcolor=sepia, urlcolor=black]{hyperref}
\usepackage{mathrsfs}
%\usepackage[utf8]{inputenc}

\newcommand{\var}{\operatorname{var}}
\newcommand{\cov}{\operatorname{cov}}


\usepackage{listings} %Per inserire codice
\lstnewenvironment{codice_c}[1][]
{\lstset{basicstyle=\small\ttfamily, columns=fullflexible,
keywordstyle=\color{red}\bfseries, commentstyle=\color{blue},
language=C, basicstyle=\small,
numbers=left, numberstyle=\tiny,
stepnumber=2, numbersep=5pt, frame=shadowbox,  showstringspaces=false, #1}}{}

\setcounter{section}{-1}

%==================================================
%                  PRIMA PAGINA
%==================================================

\title{\textsc{Studio delle leggi dei Gas}}
\author{\small{G. Galbato Muscio} \and \small{L. Gravina} \and \small{L. Graziotto}}
\date{}

\begin{document}
	\begin{figure}
		\centering
		\includegraphics[scale=0.5, trim={2.8cm 8.9cm 0 9cm}, clip]{logo.png}
	\end{figure}
	\maketitle
	\begin{center} 
		\fbox{{\fontsize{12pt}{8mm}\textsc{Gruppo B}}} \\
		\vspace{1cm}
		\begin{tabular}{ccc}
			Esperienza di laboratorio && Consegna della relazione \\
			\emph{\small{6 novembre 2017}} &&  \emph{\small{19 novembre 2017}}\\
			\emph{\small{15 novembre 2017}} &&  \\
		\end{tabular} 
		\vspace{0.5cm}
	\end{center}
\hrule
\vspace{0.5cm}
\begin{abstract}
Mediante un sistema pistone-cilindro, collegabile ad una camera di espansione, si studiano le dipendenze di volume, pressione e temperatura di un gas (l'aria) dalle altre variabili di stato, al fine di verificare la validit� delle Leggi di Boyle, Gay-Lussac e Charles. Si realizza quindi un ciclo termico.
\end{abstract}
\newpage
\tableofcontents %Indice
\listoftables %Indice delle tabelle
\listoffigures %Indice dei grafici

\pagebreak

%==================================================
%         SCOPO E DESCRIZIONE DELL'ESPERIENZA
%==================================================
\section{Scopo e descrizione dell'esperienza}
\label{sec:descrizione}


Per l'analisi dati si utilizzer� un notebook in linguaggio \emph{Python}.


%==================================================
%				APPARATO SPERIMENTALE
%==================================================		
\section{Apparato Sperimentale}
\subsection{Strumenti}
\label{subsec:strumenti}
\begin{itemize}
	\item Pistone in grafite libero di scorrere con attrito trascurabile in un cilindro in pyrex [diametro $\Phi = \SI{32.5 \pm 0.1}{mm}$];
	\item Camera di espansione metallica chiusa da tappo in gomma;
	\item Termometro a mercurio [risoluzione: \SI{0.2}{\degree C}, incertezza: \SI{0.03}{\degree C}];
	\item 2 Calorimetri Dewar;
	\item Tappo per calorimetro;
	\item Bilancia [portata: \SI{8000}{g}, risoluzione: \SI{0.1}{g}, incertezza: \SI{0.03}{g}].
\end{itemize}
\subsection{Sensori}
I seguenti sensori utilizzati sono interfacciati con il software \emph{DataStudio}.
\begin{itemize}
	\item Sensore di posizione angolare, che registra quindi lo spostamento del pistone [risoluzione: \SI{1.e-5}{m}];
	\item Sensore di bassa pressione [risoluzione: \SI{0.01}{kPa}];
	\item Sensore di temperatura [risoluzione: \SI{1.e-5}{\degree C}].
\end{itemize}
%==================================================
%            SEQUENZA OPERAZIONI SPERIMENTALI
%==================================================
\section{Sequenza Operazioni Sperimentali} 

\subsection{Costante di tempo del termometro}
\label{subsec:costante_termometro}

\subsection{Verifica della Legge di Boyle}
\label{subsec:Boyle}

\subsection{Verifica della Legge di Gay-Lussac}
\label{subsec:Gay-Lussac}

\subsection{Verifica della Legge di Charles}
\label{subsec:Charles}

\subsection{Realizzazione di un ciclo termico}
\label{subsec:ciclo}


%==================================================
%				    CONCLUSIONI
%==================================================
\section{Considerazioni finali}








\end{document}
